\chapter{Pendahuluan}

Gelombang laut merupakan fenomena alam yang terjadi di laut. Secara
umum gelombang dapat didefinisikan sebagai suatu sinyal yang
teridentifikasi, merambat dengan perantara suatu medium, mentransfer
energi dan memiliki kecepatan rambat. Penyebab terjadinya gelombang
di permukaan laut bermacam-macam, diantaranya adalah dari hembusan
angin yang memungkinkan gelombang untuk menjalar ke segala arah
sesuai dengan arah angin. Pergerakan gelombang dapat mengarah menuju
pantai maupun menjauhi pantai.


\noindent Komponen terpenting dalam mempelajari gelombang adalah
amplitudo, yaitu simpangan maksimum permukaan gelombang. Gelombang
dengan energi dan amplitudo yang cukup besar yang bergerak ke arah
pantai memiliki kemampuan untuk merusak daerah yang dilaluinya, maka
tidak tertutup kemungkinan bahwa gelombang tersebut berpotensi
menyebabkan terjadinya kerusakan di sekitar pantai. Banyak cara
diupayakan untuk mengurangi dampak gelombang tersebut. Salah satu
cara mencegah kerusakan pantai akibat besarnya amplitudo gelombang
datang tersebut yaitu dengan memanfaatkan \emph{breakwater} atau
pemecah gelombang. Keberadaan \emph{breakwater} ini berfungsi untuk
memperkecil amplitudo gelombang dan mengurangi kecepatan gelombang
sehingga energi gelombang saat menumbuk pantai sudah berkurang.
Dengan demikian dampak kerusakannya juga lebih kecil. Tugas akhir
ini membahas \emph{breakwater} berupa reef ball. Di sini tumpukan
reef ball dipandang sebagai balok berpori dengan porositas dan
permeabilitas tertentu. Selain untuk memecah gelombang, reef ball
ini juga berfungsi sebagai terumbu karang buatan, mengingat terumbu
karang yang ada di dunia ini sudah
banyak yang mengalami kerusakan.\\


\noindent Jika terdapat gelombang yang menjalar ke arah pantai, maka
setelah melalui pemecah gelombang akan ada sebagian dari gelombang
tersebut yang diteruskan ke arah pantai, dan sebagian lagi akan
dipantulkan berbalik arah. Karakteristik media berpori dapat
mempengaruhi terbentuknya gelombang permukaan. Karakteristik media
berpori yang mempengaruhinya diantaranya adalah porositas,
permeabilitas, serta ukuran dari media berpori. Gelombang yang
diteruskan ke arah pantai setelah melalui balok berpori dinamakan
gelombang transmisi. Setelah melalui \emph{breakwater} berpori
dengan karakteristik tertentu yang diharapkan dapat diperoleh
gelombang transmisi dengan amplitudo yang lebih kecil dibanding
amplitudo gelombang datangnya.  Kemampuan balok berpori meredam
amplitudo gelombang yang datang menjadi salah satu permasalahan yang
menarik untuk dikaji. Pada tugas akhir ini, akan
dilihat pola perambatan gelombang permukaan di atas media berpori.\\


\noindent Berdasarkan latar belakang yang telah diuraikan, penulis
merumuskan masalah sebagai berikut :
\begin{enumerate}
  \item Bagaimana model matematika untuk pembentukan gelombang
  permukaan di atas media berpori.
  \item Bagaimana pengaruh karakteristik dan ukuran media berpori
  dalam meredam gelombang.
\end{enumerate}


\noindent Berdasarkan latar belakang dan rumusan masalah yang telah
diuraikan maka tugas akhir ini bertujuan untuk membahas :
\begin{enumerate}
  \item Model matematika untuk pembentukan gelombang permukaan di
  atas media berpori. Akan dilihat fenomena pembentukan gelombang permukaan yang dipengaruhi
  gerakan partikel fluida melalui media berpori.
  \item Pengaruh karakteristik dan ukuran media berpori dalam
  meredam gelombang. Dengan menghitung \emph{wave transmission coefficient}
  maka kita dapat memperoleh pengaruh karakteristik dan ukuran media
  berpori dalam meredam gelombang.
\end{enumerate}


\noindent Dengan mengetahui perilaku gelombang yang mengalami
reduksi amplitudo setelah melalui pemecah gelombang, maka dapat
dibangun desain peredam gelombang berupa reef ball. Karakteristik
yang mempengaruhi peredaman gelombang antara lain porositas,
permeabilitas, dan ukuran reef ball. Setelah mengkaji karakteristik
dari reef ball maka dapat dibuat desain reef ball yang dapat
mengoptimumkan reduksi amplitudo gelombang. Diharapkan dengan ukuran
optimal \emph{breakwater} berpori berupa reef ball ini dapat
mengurangi kerusakan pada pantai akibat amplitudo gelombang
transmisi yang terlalu tinggi. Selain itu reef ball ini juga
bermanfaat sebagai
terumbu karang buatan.\\


\noindent Penelitian ini melibatkan aspek-aspek fisis dengan
menformulasikan persamaan-persamaan dasar berupa hukum kekekalan
massa dan persamaan momentum. Teori tentang dinamika fluida yang
mendasari pemodelan fenomena pembentukan gelombang permukaan di atas
media berpori. Fluida yang dikaji di sini adalah fluida tak kental,
tak termampatkan, dan alirannya irrotasional. Dalam penyusunan model
matematika, digunakan teori aliran potensial untuk fluida. Pada
setiap media berlaku hukum kekekalan massa dan persamaan momentum.
Persamaan Bernoulli dapat diperoleh berdasarkan persamaan momentum.
Persamaan Bernoulli digunakan untuk memodelkan kondisi batas dinamik
pada media fluida dan media berpori. Diasumsikan partikel fluida
yang berada di permukaan akan selalu berada di permukaan. Hal
tersebut digunakan untuk memodelkan batas kinematik. Gelombang yang
ditinjau pada tugas akhir ini adalah gelombang monokromatik. Untuk
menyederhanakan
pembahasan maka tugas akhir ini hanya akan membahas model linier.\\

\noindent Model yang telah diperoleh diselesaikan secara numerik
dengan metode Lax. Metode ini diharapkan mampu mensimulasikan
fenomena terbentuknya gelombang permukaan yang dipengaruhi gerakan
partikel fluida melalui media berpori. Selanjutnya dilakukan
perhitungan seberapa besar reduksi
amplitudo gelombang akibat adanya media berpori dengan karakteristik dan ukuran tertentu.\\


\noindent Metode yang digunakan dalam penyusunan tugas akhir ini
adalah metode deskriptif melalui studi literatur. Sebagian besar
studi literatur yang digunakan yaitu mempelajari jurnal. Selain dari
jurnal, juga didukung dengan membaca berbagai literatur lainnya
seperti buku, internet, dan artikel. Penulis juga menggunakan
software matlab untuk membuat simulasi dari model yang telah
diperoleh. Hasil simulasi model tersebut yang akan dianalisis. Hasil
dari simulasi model merupakan hasil numerik yang akan dibandingkan
dengan hasil analitik.\\


\noindent Tugas akhir ini terdiri dari 5 bab. Pada bab pertama
dipaparkan mengenai latar belakang, rumusan masalah, tujuan dan
manfaat penelitian ini, dan teknik
penelitian.\\

\noindent Pada bab dua, penulis menguraikan proses dan
langkah-langkah dalam memodelkan gelombang gelombang permukaan di
atas
breakwater berupa balok berpori, kasus dasar tak rata. Serta membahas relasi dispersi kasus dasar rata.\\

\noindent Pada bab tiga, penulis menguraikan semua proses dan
langkah-langkah menyelesaikan masalah secara numerik serta
menjelaskan metode yang digunakan. Pada bab ini akan dikaji studi kualitatif dan kuantitatif
dengan memodifikasi parameter yang ada untuk melihat pengaruh dari masing-masing parameter terhadap fenomena
reduksi amplitudo gelombang.\\

\noindent Pada bab empat, penulis membandingkan antara hasil
perhitungan numerik dan eksperimen. Bab ini akan menunjukkan
kesesuaian model dengan data eksperimen.\\

\noindent Pada bab lima akan diuraikan rangkuman hasil penelitian
dan pada akhirnya menuangkan simpulan secara keseluruhan dan saran
sebagai penutupnya.\\
