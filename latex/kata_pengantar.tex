\begin{terimakasih}
\noindent \textit{Bismillahirrahmanirrahim.}

\noindent \textit{Ahmadu rabbillaha khaira maliki. Mushalliyan 'alannabiyyil musthafa. Wa alihil mustakmilinasy syarafa.}

\noindent Puji dan syukur penulis panjatkan kepada Allah SWT Yang Maha Merajai. Melalui qadha dan qadar-Nya, penulis dapat menyelesaikan tugas akhir dengan judul \textbf{Jaringan Saraf Fuzzy untuk Menyelesaikan Masalah Klasifikasi Multikelas} sebagai salah satu syarat kelulusan dari Program Studi Sarjana Matematika Institut Teknologi Bandung.

\noindent Tugas akhir ini membahas salah satu model pembelajaran mesin, yaitu jaringan saraf fuzzy. Pada praktiknya, proses pembangunan model jaringan saraf fuzzy ini menerapkan beberapa bidang ilmu matematika, di antaranya: konsep fuzzy, optimisasi, dan statistika. Penulis mengharapkan topik tugas akhir ini dapat dikembangkan lebih jauh lagi dan hasil penelitian yang diperoleh bisa lebih baik dari hasil penelitian pada tugas akhir ini.

\noindent Berbagai pihak telah membantu penulis dalam penyelesaian tugas akhir ini, baik secara langsung maupun tidak langsung. Penulis menyebut berbagai pihak ini sebagai \textit{supporting system}. Penulis ingin memberikan apresiasi kepada \textit{supporting system} tersebut dengan cara mengabadikan nama-nama mereka di dalam tugas akhir ini. \textit{Supporting system} bagi penulis adalah sebagai berikut:

\begin{enumerate}
    \item Kedua orang tua penulis, \textit{Mamah} Ida Rosidah dan \textit{Bapa} Abdul Rohim. Orang tua yang tidak terlalu banyak menuntut kepada anak-anaknya, hanya meminta anak-anaknya supaya tidak melupakan ibadah wajib. Penulis mengucapkan terima kasih kepada \textit{Mamah} dan \textit{Bapa} atas segalanya yang telah \textit{Mamah} dan \textit{Bapa} berikan: doa orang tua yang terus menerus, bantuan morel dan materiel, serta bimbingan dalam menjalani kehidupan.
    \item Bapak Dr. Agus Yodi Gunawan sebagai dosen pembimbing Tugas Akhir penulis. Terima kasih telah menginisiasi pemahaman penulis mengenai konsep fuzzy dan telah menjadi tempat bertanya yang selalu ada jawabannya bagi penulis.
    \item Bapak Dr. Sapto Wahyu Indratno, M.Si. sebagai dosen penguji pada Seminar Tugas Akhir I penulis. Terima kasih telah membuka pemahaman penulis mengenai \textit{big data} dan pembelajaran mesin.
    \item Bapak Prof. Dr. Marcus Wono Setya Budhi dan Ibu Dr. Finny Oktariani, M.Si. sebagai dosen penguji pada Seminar Tugas Akhir II penulis. Terima kasih atas saran dan masukan yang telah Bapak dan Ibu berikan, sehingga isi tugas akhir ini bisa menjadi lebih baik dari sebelum seminar.
    \item Bapak Warsoma Djohan, M.Si. sebagai dosen yang menerima penulis menjadi asistennya untuk matakuliah Pengantar Teknologi Informasi B. Terima kasih telah menyokong penulis dalam pengerjaan tugas akhir ini melalui fasilitas yang Bapak berikan.
    \item Bapak Prof. Dr. M. Salman A.N., M.Si. sebagai dosen wali penulis. Terima kasih telah memberikan wejangan kepada penulis selama penulis belajar di Program Studi Sarjana Matematika ITB.
    \item Simpang \textit{Flat Earth}: Nyahmet, Rangga, Haris, Dancent, Firli, Febi, Andre, dan Arjun. Terima kasih telah bersama-sama dalam suka dan duka selama tiga tahun ini. Meminjam perkataan dari Nyahmet: semoga Tuhan mengampuni kita semua.
    \item Teman-teman penulis yang cukup sering menghabiskan waktu bersama di lab: Adnan, Rizka, Aul, Manda, Clarissa, Fadhel, dan Raka. Terima kasih telah saling menemani dalam proses pengerjaan tugas akhir.
    \item Semua teman-teman FRACTAL Matematika ITB 2015. Terima kasih telah bersama-sama menjalani perkuliahan di Matematika ITB.
    \item HIMATIKA ITB, terutama Badan Pengurus HIMATIKA ITB 2018/2019, Tim Futsal HIMATIKA ITB, Pendiklat Sekolah Komandan Lapangan (Sekdan) HIMATIKA ITB 2017, dan teman-tema sesama peserta sekdan. Terima kasih telah memberikan ruang bagi penulis untuk aktualisasi diri dan menginisiasi penulis untuk terus berproses menjadi manusia seutuhnya.
    \item Adik penulis, Mohammad Rizky Zakary. Terima kasih telah menemani penulis pada saat jenuh dari proses penulisan tugas akhir ini. Semoga kita bisa terus membahagiakan \textit{Mamah} dan \textit{Bapa}, \textit{De}.
\end{enumerate}

\noindent Penulis menyadari bahwa tugas akhir ini masih jauh dari kata sempurna. Oleh karena itu, penulis akan sangat berterima kasih jika ada yang menyampaikan kritik dan saran mengenai tugas akhir ini. Semoga tugas akhir ini dapat bermanfaat bagi pembaca. Terima kasih

\flushright
Bandung, Juni 2019\\
Penulis
\end{terimakasih}