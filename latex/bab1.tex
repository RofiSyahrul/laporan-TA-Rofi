\vspace{15cm}
\chapter{Pendahuluan} \label{bab satu}

\noindent Pembelajaran mesin atau \emph{machine learning} (\gls{ml}), sebagaimana didefinisikan oleh \citeasnoun{mohri} sebagai sarana yang memungkinkan komputer untuk memanfaatkan informasi masa lalu dengan tujuan meningkatkan kinerja atau membuat prediksi yang akurat, telah menjadi sarana yang tepat dan menjanjikan di era revolusi industri 4.0. \citeasnoun{wuest} telah berpendapat dari perspektif manufaktur mengenai alasan ML menjadi sarana yang ampuh di era ini. Tantangan utama di era revolusi industri 4.0 adalah terjadinya peningkatan kompleksitas, kedinamisan, dan ketersediaan data. ML dapat mengatasi tantangan-tantangan tersebut karena sarana ini dapat mengatasi masalah dan dataset yang berdimensi tinggi dengan upaya yang masuk akal. ML juga dapat memperoleh pola dari data yang ada dan mendapatkan perkiraan perilaku masa depan \cite{adin}. Informasi baru ini dapat mendukung pemilik proses dalam pengambilan keputusan atau digunakan untuk meningkatkan sistem secara otomatis \cite{wuest}.

\noindent Salah satu masalah yang dapat ditangani oleh algoritma ML adalah masalah klasifikasi multikelas. Masalah klasifikasi multikelas ini termasuk tugas dari \gls{sml} (\emph{supervised machine learning}). SML merupakan skenario ML yang dilakukan terhadap data yang setiap observasinya memiliki label tertentu \cite{mohri}. Pada masalah klasifikasi multikelas, terdapat lebih dari dua label yang berbeda dari semua sampel data, dan setiap labelnya berupa kategori, bukan bilangan real. Data yang hanya memiliki dua label kategori yang berbeda merupakan masalah klasifikasi biner. Model ML yang digunakan untuk menyelesaikan masalah klasifikasi biner di antaranya adalah model pengklasifikasi bayes, model regresi logistik, \emph{k-nearest neighbours}, \emph{support vector machine}, dan lain-lain \cite{rogers}. Model-model ini dapat diperluas untuk menyelesaikan masalah klasifikasi multikelas dengan cara menggunakan skema pengkodean tertentu terhadap setiap labelnya. Skema pengkodean yang biasa digunakan pada klasifikasi multikelas yaitu skema satu lawan semua, skema satu lawan satu, dan skema satu lawan orde yang lebih tinggi \cite{ou}.

\noindent Model Jaringan Saraf Tiruan (\gls{jst}), sebagaimana didefinisikan oleh \citeasnoun{kriesel} dan \citeasnoun{dgrau} sebagai jaringan komputasi yang terdiri dari koneksi yang memiliki nilai bobot antar-neuron dengan tujuan untuk mensimulasikan proses pengambilan keputusan dalam neuron dari sistem saraf biologis, merupakan model pembelajaran mesin yang sangat efisien untuk menangani data yang berukuran sangat besar. Model JST dapat digunakan untuk menyelesaikan masalah klasifikasi multikelas seperti yang telah dilakukan oleh \citeasnoun{krai} dan \citeasnoun{ou}. Model JST sangat sederhana secara komputasi dan algoritma, memiliki fitur pengorganisasian tersendiri yang memungkinkannya menangani berbagai masalah, dan memiliki tingkat keparalelan yang tinggi \cite{dgrau}. JST bekerja dengan cara melakukan transformasi terhadap informasi yang ada secara berlapis-lapis karena JST memiliki beberapa lapisan tersembunyi. Akibatnya, JST dapat mengurangi dan menghapus kontaminasi pada data mentah, sehingga JST berguna dalam pengurangan gangguan \cite{deng}. \citeasnoun{schmid} telah merangkum bahwa model ini juga berhasil mengantarkan penggunanya memenangkan berbagai kompetisi. JST dengan garis tunda internal pada kekacauan getaran intensitas dari laser NH3 memenangkan kompetisi \emph{Santa Fe time-series}. JST Bayes berdasarkan ansambel jaringan saraf memenangkan \emph{NIPS 2003 Feature Selection Challenge}. Sebuah sistem yang memuat pertumbuhan JST telah memenangkan kontes \emph{CASP 2012} pada prediksi peta kontak protein.

\noindent Sebagaimana telah disebutkan sebelumnya bahwa di era revolusi industri 4.0 jumlah ketersediaan data meningkat secara signifikan. Tetapi, data besar ini mungkin memuat sejumlah gangguan yang tinggi dan ketidakpastian yang tidak dapat diprediksi \cite{deng}. Jaringan saraf tiruan, sebagaimana telah dijelaskan di paragraf sebelumnya, mampu mengurangi gangguan. Untuk memasukkan unsur ketidakpastian, dibutuhkan model ML yang menggunakan konsep logika fuzzy. Hal ini dikarenakan nilai kebenaran dari suatu pernyataan logika fuzzy bersifat tidak pasti antara benar dan salah \cite{zadeh}. Lebih jauh lagi, logika fuzzy dapat mengatasi ketidakpastian pada data mentah dengan cara membangun aturan fuzzy secara fleksibel \cite{deng}.

\noindent Dengan demikian, dibutuhkan model ML yang dapat mengatasi kelemahan dari data besar. Model ML yang dibutuhkan ini adalah gabungan antara JST dengan konsep logika fuzzy yang disebut dengan model jaringan saraf fuzzy. Model jaringan saraf fuzzy ini telah digunakan oleh \citeasnoun{lee} dan \citeasnoun{yeh} pada SML yang bukan masalah klasifikasi. Masalah klasifikasi multikelas juga dapat diselesaikan menggunakan model jaringan saraf fuzzy seperti yang telah dilakukan oleh \citeasnoun{ghongade} dan \citeasnoun{rangkuti}.

\noindent Berdasarkan latar belakang di atas, tujuan tugas akhir ini adalah sebagai berikut.
\begin{itemize}
    \item Merancang program yang dapat mengimplementasikan model dan skema ML. Program ini terdiri dari tiga skema klasifikasi multikelas. Setiap skema klasifikasi menggunakan model ML berupa jaringan saraf fuzzy. Program ini dibuat untuk menyelesaikan masalah klasifikasi multikelas.
    \item Menguji model dan skema ML dengan cara membandingkan akurasi dari tiga skema klasifikasi multikelas untuk setiap data yang menjadi objek pengujian.
\end{itemize}

\noindent Metode yang digunakan dalam penyusunan tugas akhir ini adalah metode deskriptif melalui studi literatur. Studi literatur yang digunakan yaitu dengan cara mempelajari jurnal dan buku. Selain itu, dilakukan juga proses pembangunan model jaringan saraf fuzzy beserta skema ML tertentu berdasarkan studi literatur. Kemudian model jaringan fuzzy dan skemanya ini diimplementasikan ke dalam program. Penulis merancang program dengan bentuk \emph{Graphical User Interface} (\gls{gui}) supaya memudahkan dalam proses pembangunan model jaringan saraf fuzzy serta pengujian model dan skemanya. Program ini dirancang menggunakan bahasa pemrograman python versi 3.6.4 dengan bantuan beberapa \emph{package}, di antaranya: \emph{numpy}, \emph{pandas}, \emph{scikit-learn}, dan \emph{tkinter}.

\noindent Dengan adanya program yang dirancang melalui tugas akhir ini, maka telah terbentuk program baru yang nantinya dapat digunakan untuk menyelesaikan masalah klasifikasi multikelas dengan model jaringan saraf fuzzy pada data lain di dunia nyata. Jika diperoleh akurasi yang memuaskan, maka akan menambah keyakinan dalam menggunakan program ini untuk membangun dan menguji model jaringan saraf fuzzy dari data lain. Program yang dibuat ini berbentuk GUI, sehingga program ini dapat digunakan oleh siapapun tanpa harus melakukan instalasi python terlebih dahulu.

\noindent Supaya data dapat digunakan untuk membangun serta menguji model dan skema ML yang akan dibangun, maka data tersebut harus memenuhi syarat-syarat berikut ini.
\begin{itemize}
    \item Data harus terdiri dari bagian masukan dan bagian keluaran. Bagian masukan dapat memuat beberapa kolom yang setiap entrinya dapat berupa bilangan real ataupun kategori. Bagian keluaran hanya memuat satu kolom yang setiap entrinya berupa kategori, bukan bilangan real. Banyaknya kategori yang berbeda pada seluruh bagian keluaran harus lebih dari dua. Hanya terdapat satu kategori yang termuat pada setiap observasi dalam data keluaran.
    \item Data tidak memuat entri dengan nilai yang kosong.
\end{itemize}
Semua data yang akan digunakan ini diasumsikan memiliki kolom-kolom pada bagian masukan yang saling bebas. Selain itu, diasumsikan juga setiap sampel pada data keluaran bergantung kepada semua data masukan dari sampel tersebut. Data yang digunakan juga diasumsikan tidak memiliki pencilan.

\noindent Tugas akhir ini terdiri dari 5 bab. Bab \ref{bab satu} memaparkan latar belakang, tujuan penelitian, metode penelitian, manfaat penelitian, batasan masalah, dan asumsi yang digunakan.

\noindent Pada Bab \ref{bab jsf}, penulis mengelaborasi model ML yang digunakan, yaitu jaringan saraf fuzzy. Pembahasannya dimulai dari konsep dasar fuzzy.

\noindent Pada Bab \ref{bab skema}, penulis membahas skema ML yang digunakan untuk menyelesaikan masalah klasifikasi multikelas. Skema ML yang dibahas meliputi pra pemrosesan data, skema klasifikasi multikelas, dan skema pemilihan model.

\noindent Pada Bab \ref{bab uji}, penulis menampilkan dan membahas hasil pengujian model dan skema ML terhadap empat data. 

\noindent Pada Bab \ref{bab konklusi}, akan diuraikan rangkuman hasil penelitian secara keseluruhan dan ditutup dengan saran terkait penelitian ini.