\begin{Abstrak} % 500 - 800 kata
Konstruksi program pembelajaran mesin yang dapat menyelesaikan masalah klasifikasi multikelas dipelajari pada tugas akhir ini. Jaringan saraf fuzzy, yaitu jaringan saraf tiruan yang proses \emph{feed-forward}-nya berupa tahap-tahap dari sistem kontrol logika fuzzy yang dibangun dari data masukan dan keluaran yang tersedia, akan digunakan sebagai model pembelajaran mesin. Tahap-tahap tersebut adalah tahap fuzzifikasi, inferensi fuzzy, dan defuzzifikasi. Tahap inferensi fuzzy membutuhkan aturan fuzzy dan hasil dari tahap fuzzifikasi. Aturan fuzzy terdiri dari beberapa implikasi fuzzy yang saling berkaitan. Untuk mendapatkan model jaringan saraf fuzzy, dibutuhkan dua fase: identifikasi struktur dan identifikasi parameter. Fase identifikasi struktur menghasilkan struktur awal dari aturan fuzzy yang meliputi banyaknya implikasi pada aturan fuzzy dan nilai awal dari setiap parameter yang terlibat di dalam model jaringan saraf fuzzy. Fase identifikasi parameter melakukan perbaikan nilai-nilai dari parameter yang terlibat di dalam model. Untuk mendapatkan model jaringan saraf fuzzy yang paling optimal, diperlukan proses validasi silang secara berulang. Proses validasi silang menghasilkan rata-rata akurasi untuk suatu model. Model dengan proses validasi silang yang menghasilkan rata-rata akurasi terbesar merupakan model yang paling optimal. Untuk tugas akhir ini, dilakukan simulasi terhadap tiga jenis data, yaitu: data koordinat kartesius, data tanaman iris, dan data evaluasi mobil. Akan diterapkan tiga skema klasifikasi multikelas, yaitu: satu lawan semua, satu lawan satu, dan satu lawan orde yang lebih tinggi. Model jaringan saraf fuzzy yang diperoleh dengan skema klasifikasi satu lawan semua mempunyai akurasi lebih dari $95\%$. Model jaringan saraf fuzzy dengan skema klasifikasi satu lawan satu mempunyai akurasi setidaknya sebesar $94\%$. Model jaringan saraf fuzzy dengan skema klasifikasi satu lawan orde yang lebih tinggi mempunyai akurasi lebih dari $84\%$.

\katakunci{pembelajaran mesin, jaringan saraf fuzzy, identifikasi struktur, identifikasi parameter, klasifikasi multikelas, validasi silang}
\end{Abstrak}
%-------------------------------------
\begin{Abstract}
A construction of machine learning programs that can solve multi-class classification problems is studied in this final project. Fuzzy neural network, i.e. an artificial neural network whose feed-forward process is the stages of fuzzy logic control system that is constructed based on available input and output data, will be used as machine learning models. The stages are fuzzification, fuzzy inference, and defuzzification. The stage of fuzzy inference requires fuzzy rules and the results of fuzzification stage. Fuzzy rules consist of several interrelated fuzzy implications. Two phases are required to get the fuzzy neural network model: structure identification and parameter identification. The structure identification phase produces the initial structure of fuzzy rules which includes number of implications for fuzzy rules and the initial values of each parameter involved in the fuzzy neural network model. The parameter identification phase refines the values of the parameters involved in the model. To get the most optimal fuzzy neural network model, a cross-validation process is needed repeatedly. The cross validation process produces an average accuracy for a model. The model with a cross validation process that produces the highest average accuracy is the most optimal model. For this final project, three types of data are simulated, i.e.: cartesian coordinate data, iris plant data, and car evaluation data. Three schemes of multi-class classification are applied: one against all, one against one, and one against higher order. The fuzzy neural network model obtained with a one against all classification scheme has an accuracy of more than $95\%$. The fuzzy neural network model obtained with a one against one classification scheme has an accuracy of at least $94\%$. The fuzzy neural network model obtained with a one against higher order scheme has an accuracy of more than $84\%$.

\keywords{machine learning, fuzzy neural network, structure identification, parameter identification, multi-class classification, cross validation}

\end{Abstract}


%Abstrak terdiri atas 500 - 800 kata dan memuat permasalahan yang dikaji, metode
%yang digunakan, ulasan singkat serta
%penjelasan hasil penelitian dan kesimpulan yang diperoleh dan kontribusi kandidat
%doktor dalam perkembangan keilmuan yang dikaji. Di dalam abstrak tidak boleh
%ada referensi 

%Pada tahap fuzzifikasi, serangkaian fakta yang berupa bilangan real dikonversi menjadi nilai-nilai yang berada pada selang $[0,1]$ menggunakan fungsi keanggotaan himpunan fuzzy yang bersesuaian. Pada tugas akhir ini, model jaringan saraf fuzzy hanya menggunakan fungsi gauss sebagai fungsi keanggotaan himpunan fuzzy. 
%Tahap inferensi fuzzy membutuhkan aturan fuzzy dan hasil dari tahap fuzzifikasi. Aturan fuzzy terdiri dari beberapa implikasi fuzzy yang saling berkaitan. %Implikasi fuzzy terdiri dari bagian pendahulu dan bagian konsekuensi. Pada tahap inferensi fuzzy, akan dihasilkan kesimpulan yang berupa himpunan fuzzy dan direpresentasikan oleh fungsi keanggotaannya. 
%Pada tahap defuzzifikasi, himpunan fuzzy tersebut akan dikonversi menjadi bilangan real yang sesuai.
%Model jaringan saraf fuzzy dibangun dari data yang diberikan. Untuk mendapatkan model jaringan saraf fuzzy, dibutuhkan fase identifikasi struktur dan identifikasi parameter. Fase identifikasi struktur menghasilkan struktur awal dari aturan fuzzy yang meliputi banyaknya implikasi pada aturan fuzzy dan nilai awal dari setiap parameter yang terlibat di dalam model jaringan saraf fuzzy. Fase identifikasi parameter melakukan perbaikan nilai-nilai dari parameter yang terlibat di dalam model. 
%Data pada klasifikasi multikelas memiliki lebih dari dua kategori pada seluruh labelnya. Kategori pada label akan dikonversi menjadi bilangan real melalui skema satu lawan satu, skema satu lawan semua, atau skema satu lawan orde yang lebih tinggi. Untuk mendapatkan model jaringan saraf fuzzy yang optimal, diperlukan proses validasi silang secara berulang. Proses validasi silang menghasilkan rata-rata akurasi untuk suatu model. Model dengan proses validasi silang yang menghasilkan rata-rata akurasi terbesar merupakan model yang paling optimal. Pada tugas akhir ini, diperoleh 12 model jaringan saraf fuzzy yang dibangun dari data koordinat I, data koordinat II, data tanaman iris, dan data evaluasi mobil menggunakan tiga skema klasifikasi multikelas. Model jaringan saraf fuzzy yang diperoleh dengan skema klasifikasi satu lawan semua mempunyai akurasi lebih dari $95\%$. Model jaringan saraf fuzzy dengan skema klasifikasi satu lawan satu mempunyai akurasi setidaknya sebesar $94\%$. Model jaringan saraf fuzzy dengan skema klasifikasi satu lawan orde yang lebih tinggi mempunyai akurasi lebih dari $84\%$. Dengan demikian, dapat disimpulkan bahwa jaringan saraf fuzzy dapat digunakan untuk menyelesaikan masalah klasifikasi multikelas.