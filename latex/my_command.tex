\usepackage{colortbl}  %--- untuk membuat tabel berwarna
\usepackage{listings}  %--- untuk menulis listing program
\usepackage{xcolor}    %--- memanfaatkan warna LaTeX
\usepackage{caption}
\usepackage{color}
\usepackage{lscape}
\usepackage{amsmath}
\usepackage{amsfonts}
\usepackage{commath}
\usepackage{chemarr}
\usepackage{nomencl}
\usepackage{etoolbox}
% \usepackage{eufrak}
% \usepackage{arrayjob}
% \usepackage{arrayjobx}

\usepackage{hyperref}
\urlstyle{rm}
\renewcommand\UrlFont{\rmfamily\itshape}
% \hypersetup{
%     pdftitle={Draft Tugas Akhir - Syahrul Rofi - 10115077},
%     bookmarks=true,
%     }
%\usepackage[printwatermark]{xwatermark}
%\usepackage{draftwatermark} %\usepackage[firstpage]{draftwatermark}

\usepackage{booktabs}  % untuk tabel
\usepackage{longtable}
%\usepackage{tabularx}
\usepackage{tabu}
%\usepackage{subfigure}
\usepackage{subcaption}
\usepackage{multicol}
\usepackage{multirow}
%\usepackage{sectsty}
%\usepackage{lipsum}
\usepackage{graphicx}
%\usepackage{latexsym}
%\usepackage{float}
\usepackage{array}
%\usepackage{mathrsfs}%
%\usepackage{lineno}
\usepackage{relsize} %modifikasi ukuran operator matematika
\usepackage{amssymb}
\usepackage[decimalsymbol=comma]{siunitx} %dapat mengganti simbol desimal dengan koma

\usepackage{tikz}
\usepackage{mathtools}
\usepackage{pgfplots}
\pgfplotsset{compat=1.15}

\usepackage{pgffor}
\usepackage{xparse}
\usepackage{keyval}
\usepackage{pgfmath}

\usepackage[toc]{appendix}
\usepackage{fancyhdr}
%%%%%%%%%%%%%%%%%%%%%%%%%%%%%%%%%%%%%%%%%%%%%%%%%%%%%%%%%%%%%
\definecolor{abu}{gray}{0.80}  % = hitam 20%

\newcommand{\cca}{\cellcolor{abu}}
\newcommand{\ccb}{\bfseries}

\newcolumntype{L}[1]{>{\raggedright\let\newline\\\arraybackslash\hspace{0pt}}m{#1}}
\newcolumntype{P}[1]{>{\raggedright\let\newline\\\arraybackslash\hspace{0pt}}p{#1}}
\newcolumntype{C}[1]{>{\centering\let\newline\\\arraybackslash\hspace{0pt}}m{#1}}
\newcolumntype{Q}[1]{>{\centering\let\newline\\\arraybackslash\hspace{0pt}}p{#1}}
\newcolumntype{R}[1]{>{\raggedleft\let\newline\\\arraybackslash\hspace{0pt}}m{#1}}


\graphicspath{{FolderGambar/}}%----folder tempat gambar-gambar---

\newcommand{\tmatriks}[1]{%
    \bigl(\begin{smallmatrix}#1\end{smallmatrix}\bigr)}

\newtheoremstyle{teoremaku}% --- style untuk teorema dkk
    {4ex}
    {3ex}
    {\itshape}
    {}
    {\bfseries}
    {}
    {1em}
    {}

\newtheoremstyle{contohku}% --- style untuk contoh
    {5ex}
    {3ex}
    {}
    {}
    {\bfseries}
    {}
    {1em}
    {}

\theoremstyle{teoremaku}
\newtheorem{theorem}{Teorema}%[chapter]
\newtheorem{lemma}{Lema}%[chapter]
\newtheorem{conjecture}{Dugaan}[chapter]
\newtheorem{corollary}{Akibat}%[chapter]
\newtheorem{definition}{Definisi}%[chapter]
\newtheorem{open problem}{Masalah Terbuka}[chapter]
\newtheorem{assertion}{Assertion}[chapter]
\newtheorem{prop}{Proposisi}[chapter]
\newtheorem{observation}{Observasi}[chapter]
\newcommand{\ebox}{\ \ $\Box$\vspace{1ex}}
\newcommand{\pr}{{\textbf{Bukti.}}\hspace{0.3cm}}

\theoremstyle{contohku}
\newtheorem{contoh}{Contoh}[chapter]

\newcommand{\maks}{\displaystyle\mathrm{maks}}
\DeclareMathOperator{\tr}{trace}
\DeclareMathOperator{\rank}{rank}
\DeclareMathOperator{\round}{round}

\def\ifEqString#1#2{\def\testa{#1}\def\testb{#2}
    \ifx\testa\testb}

\usetikzlibrary{shapes,shadows,calc}
\usepgflibrary{arrows}
\usetikzlibrary{decorations.shapes}
\usepackage{varwidth}

\tikzset{decorates/.style=
{decorate,decoration={shape backgrounds,shape=circle,shape size=0.35mm,shape sep=2mm}},%
decorateb/.style=
{decorate,decoration={shape backgrounds,shape=circle,shape size=1mm,shape sep=3mm}}
}

%-----Boxes & circle
%-----#1 height, #2 width, #3 name of the node, #4 coordinate
\def\kotakbesar[#1,#2,#3,#4]{%
  \node[draw, rectangle, minimum height=#1, minimum width=#2, rounded corners=3] (#3) at #4 {}; %
}
%-----#1 name of the node, #2 coordinate, #3 label
\def\kotak[#1,#2]#3;{
  \node[draw,rectangle,color=black,rounded corners=3,align=center] (#1) at #2 {#3};
}
\def\bulat[#1,#2]#3;{
  \node[draw,circle,color=black,minimum size=0.8cm,inner sep=0pt,align=center] (#1) at #2 {#3};
}
\def\kotakw[#1,#2]#3;{
  \node[draw,rectangle,color=black,rounded corners=3,align=center, inner sep=2pt,minimum size=0.8cm] (#1) at #2 {\begin{varwidth}{16em}#3\end{varwidth}};
 }
\def\kotakww[#1,#2]#3;{
  \node[draw,rectangle,color=black,rounded corners=3,align=center, inner sep=2pt,minimum size=0.8cm] (#1) at #2 {\begin{varwidth}{10em}#3\end{varwidth}};
 }
\def\kosong[#1,#2]#3;{
  \node (#1) at #2 {\begin{varwidth}{11em}#3\end{varwidth}};
 }%[font=\fontsize{#3}{0}\selectfont]
%-----#3 & #4 label
\def\kotaktl[#1,#2]#3#4;{
  \node[draw,rectangle,color=black,rounded corners=3,align=center] (#1) at #2 {#3\\#4};
}
\def\bulattl[#1,#2]#3#4;{
  \node[draw,circle,color=black,minimum size=0.8cm,inner sep=0pt,align=center] (#1) at #2 {#3\\#4};
}
%-----#1 from node, #2 to node, #3 angle of a node, #4 position of a node, #5 node label
%-----dashed, or other parameters for draw
\def\singleT[#1,#2,#3,#4]#5;{ 
  \draw[-triangle 60,color=black!20!black] (#1) -- (#2) node[midway,sloped,#4,rotate=#3,font=\fontsize{8}{0}\selectfont,text width=2.3cm, align=center] {#5};  
}
\def\singleTf[#1,#2,#3,#4,#5]#6;{ 
  \draw[-triangle 60,color=black!20!black] (#1) -- (#2) node[midway,sloped,#4,rotate=#3,font=\fontsize{#5}{0}\selectfont,text width=2.3cm, align=center] {#6};
}
%%#6 text width
\def\singleTfw[#1,#2,#3,#4,#5,#6]#7;{ 
  \draw[-triangle 60,color=black!20!black] (#1) -- (#2) node[midway,sloped,#4,rotate=#3,font=\fontsize{#5}{0}\selectfont,text width=#6, align=center] {#7};
}
\def\doubleT[#1,#2,#3,#4]#5;{ 
  \draw[triangle 60-triangle 60,color=black!20!black] (#1) -- (#2) node[midway,sloped,#4,rotate=#3,font=\fontsize{8}{0}\selectfont] {#5};  
}
\def\titiks[#1,#2];{ 
  \draw[decorates,fill] (#1) -- (#2);
}
\def\titikb[#1,#2];{ 
  \draw[decorateb,fill] (#1) -- (#2);  
}
%---#1 antecedent variable, #2 antecedent's linguistic value, #3 consequent variable, #4 consequent's linguistic value
%---#5 number of antecedent variables, #6 number of fuzzy rules
\makeatletter

\newtoggle{onerule} %
\newtoggle{tworules} %
\newtoggle{threerules}%
\newtoggle{oneante}%
\newtoggle{twoante}%

\def\setboolflc#1#2{%
\ifEqString{#2}{1}%
    \global\toggletrue{onerule}%
    \global\togglefalse{tworules}%
    \global\togglefalse{threerules}%
\else \ifEqString{#2}{2}%
    \global\toggletrue{tworules}%
    \global\togglefalse{onerule}%
    \global\togglefalse{threerules}%
\else \ifEqString{#2}{3}%
    \global\toggletrue{threerules}%
    \global\togglefalse{onerule}%
    \global\togglefalse{tworules}%
\else %
    \global\togglefalse{threerules}%
    \global\togglefalse{onerule}%
    \global\togglefalse{tworules}%
\fi %

\ifEqString{#1}{1} %
    \global\toggletrue{oneante}%
    \global\togglefalse{twoante}%
\else \ifEqString{#1}{2}%
    \global\toggletrue{twoante}%
    \global\togglefalse{oneante}%
\else %
    \global\togglefalse{oneante}%
    \global\togglefalse{twoante}%
\fi %
}%
%

%---- ZADEH FUZZY RULES
\NewDocumentCommand \generalFrZadeh {O{x}O{A}O{y}O{B}O{n}O{m}}{
\setboolflc{#5}{#6}%
\vspace*{-\baselineskip}
\iftoggle{onerule}{\iftoggle{oneante}{%
        \vspace*{-\baselineskip}
        \begin{alignat*}{4}%
        \Re &: \text{ Jika } &&#1 \text{ adalah } #2 \text{,} &&\text{maka } &&#3 \text{ adalah } #4
        \end{alignat*}%
    }{\iftoggle{twoante}{%
    \vspace*{-\baselineskip}
        \begin{align*}%
            \Re &: \text{ Jika } &&#1_1 \text{ adalah } #2_1 &&\text{ dan } &&#1_2 \text{ adalah } #2_2\text{,}%
            &&\text{maka } &&#3 \text{ adalah } #4
        \end{align*}%
    }{\vspace*{-\baselineskip}
    \begin{align*}%
    \Re &: \text{ Jika } &&#1_1 \text{ adalah } #2_1 \text{,} &&\ldots \text{,} &&#1_#5 \text{ adalah } #2_#5,%
    &&\text{maka } &&#3 \text{ adalah } #4
    \end{align*}%
    }%
    }%
}{\iftoggle{tworules}{\iftoggle{oneante}{%
    \vspace*{-\baselineskip}
        \begin{align*}%
        \Re_1 &: \text{ Jika } &&#1 \text{ adalah } #2_1 \text{,} &&\text{maka } &&#3 \text{ adalah } #4_1\\%
        \Re_2 &: \text{ Jika } &&#1 \text{ adalah } #2_2 \text{,} &&\text{maka } &&#3 \text{ adalah } #4_2
        \end{align*}%
    }{\iftoggle{twoante}{%
    \vspace*{-\baselineskip}
        \begin{align*}%
        \Re_1 &: \text{ Jika } &&#1_1 \text{ adalah } #2_{1,1} &&\text{dan} &&#1_2 \text{ adalah } #2_{1,2},%
        &&\text{maka } &&#3 \text{ adalah } #4_1\\%
        \Re_2 &: \text{ Jika } &&#1_1 \text{ adalah } #2_{2,1} &&\text{dan} &&#1_2 \text{ adalah } #2_{2,2},%
        &&\text{maka } &&#3 \text{ adalah } #4_2
        \end{align*}%
    }{\vspace*{-\baselineskip}
        \begin{align*}%
        \Re_1 &: \text{ Jika } &&#1_1 \text{ adalah } #2_{1,1} \text{,} &&\ldots \text{,} &&#1_#5 \text{ adalah } #2_{1,#5} \text{,}
        &&\text{maka } &&#3 \text{ adalah } #4_1\\%
        \Re_2 &: \text{ Jika } &&#1_1 \text{ adalah } #2_{2,1} \text{,} &&\ldots \text{,} &&#1_#5 \text{ adalah } #2_{2,#5} \text{,}
        &&\text{maka } &&#3 \text{ adalah } #4_2
        \end{align*}%
    }%
    }%
}{\iftoggle{threerules}{\iftoggle{oneante}{%
        \vspace*{-\baselineskip}
        \begin{align*}%
        \Re_1 &: \text{ Jika } &&#1 \text{ adalah } #2_1 \text{,} &&\text{maka } &&#3 \text{ adalah } #4_1\\%
        \Re_2 &: \text{ Jika } &&#1 \text{ adalah } #2_2 \text{,} &&\text{maka } &&#3 \text{ adalah } #4_2\\%
        \Re_#6 &: \text{ Jika } &&#1 \text{ adalah } #2_#6 \text{,} &&\text{maka } &&#3 \text{ adalah } #4_#6
        \end{align*}%
    }{\iftoggle{twoante}{%
    \vspace*{-\baselineskip}
        \begin{align*}%
        \Re_1 &: \text{ Jika } &&#1_1 \text{ adalah } #2_{1,1} &&\text{dan} &&#1_2 \text{ adalah } #2_{1,2},%
        &&\text{maka } &&#3 \text{ adalah } #4_1\\%
        \Re_2 &: \text{ Jika } &&#1_1 \text{ adalah } #2_{2,1} &&\text{dan} &&#1_2 \text{ adalah } #2_{2,2},%
        &&\text{maka } &&#3 \text{ adalah } #4_2\\%
        \Re_#6 &: \text{ Jika } &&#1_1 \text{ adalah } #2_{#6,1} &&\text{dan} &&#1_2 \text{ adalah } #2_{#6,2},%
        &&\text{maka } &&#3 \text{ adalah } #4_#6
        \end{align*}%
    }{\vspace*{-\baselineskip}
        \begin{align*}%
        \Re_1 &: \text{ Jika } &&#1_1 \text{ adalah } #2_{1,1} \text{,} &&\ldots \text{,} &&#1_#5 \text{ adalah } #2_{1,#5} \text{,}
        &&\text{maka } &&#3 \text{ adalah } #4_1\\%
        \Re_2 &: \text{ Jika } &&#1_1 \text{ adalah } #2_{2,1} \text{,} &&\ldots \text{,} &&#1_#5 \text{ adalah } #2_{2,#5} \text{,}
        &&\text{maka } &&#3 \text{ adalah } #4_2\\%
        \Re_#6 &: \text{ Jika } &&#1_1 \text{ adalah } #2_{#6,1} \text{,} &&\ldots \text{,} &&#1_#5 \text{ adalah } #2_{#6,#5} \text{,}
        &&\text{maka } &&#3 \text{ adalah } #4_#6
        \end{align*}%
    }%
    }%
}{\iftoggle{oneante}{%
\vspace*{-\baselineskip}
        \begin{align*}%
        \Re_1 &: \text{ Jika } &&#1 \text{ adalah } #2_1 \text{,} &&\text{maka } &&#3 \text{ adalah } #4_1\\%
        \Re_2 &: \text{ Jika } &&#1 \text{ adalah } #2_2 \text{,} &&\text{maka } &&#3 \text{ adalah } #4_2\\%
        & &&\vdots \\
        \Re_#6 &: \text{ Jika } &&#1 \text{ adalah } #2_#6 \text{,} &&\text{maka } &&#3 \text{ adalah } #4_#6
        \end{align*}%
    }{\iftoggle{twoante}{%
    \vspace*{-\baselineskip}
        \begin{align*}%
        \Re_1 &: \text{ Jika } &&#1_1 \text{ adalah } #2_{1,1} &&\text{dan} &&#1_2 \text{ adalah } #2_{1,2},%
        &&\text{maka } &&#3 \text{ adalah } #4_1\\%
        \Re_2 &: \text{ Jika } &&#1_1 \text{ adalah } #2_{2,1} &&\text{dan} &&#1_2 \text{ adalah } #2_{2,2},%
        &&\text{maka } &&#3 \text{ adalah } #4_2\\%
        & && &&\vdots \\
        \Re_#6 &: \text{ Jika } &&#1_1 \text{ adalah } #2_{#6,1} &&\text{dan} &&#1_2 \text{ adalah } #2_{#6,2},%
        &&\text{maka } &&#3 \text{ adalah } #4_#6
        \end{align*}%
    }{\vspace*{-\baselineskip}
        \begin{align*}%
        \Re_1 &: \text{ Jika } &&#1_1 \text{ anggota } #2_{1,1} \text{,} &&\ldots \text{,} &&#1_#5 \text{ anggota } #2_{1,#5} \text{,}
        &&\text{maka } &&#3 \text{ anggota } #4_1\\%
        \Re_2 &: \text{ Jika } &&#1_1 \text{ anggota } #2_{2,1} \text{,} &&\ldots \text{,} &&#1_#5 \text{ anggota } #2_{2,#5} \text{,}
        &&\text{maka } &&#3 \text{ anggota } #4_2\\%
        & && &&\vdots \\
        \Re_#6 &: \text{ Jika } &&#1_1 \text{ anggota } #2_{#6,1} \text{,} &&\ldots \text{,} &&#1_#5 \text{ anggota } #2_{#6,#5} \text{,}
        &&\text{maka } &&#3 \text{ anggota } #4_#6
        \end{align*}%
    }%
    }%
}%
}%
}%
}%

%---TSK FUZZY RULES
\NewDocumentCommand\generalFrTSK{O{x}O{A}O{y}O{b}O{n}O{m}}{%
\setboolflc{#5}{#6}%
\vspace*{-\baselineskip}
\iftoggle{onerule}{\iftoggle{oneante}{%
        \vspace*{-\baselineskip}
        \begin{alignat*}{4}%
        \Re &: \text{ Jika } &&#1 \text{ adalah } #2 \text{,} &&\text{maka } &&#3 = #4_0 + #4_1#1%
        \end{alignat*}%
    }{\iftoggle{twoante}{%
    \vspace*{-\baselineskip}
        \begin{align*}%
            \Re &: \text{ Jika } &&#1_1 \text{ adalah } #2_1 &&\text{ dan } &&#1_2 \text{ adalah } #2_2\text{,}%
            &&\text{maka } &&#3 = \Vec{#4}\cdot \Vec{#1}%
        \end{align*}%
    }{\vspace*{-\baselineskip}
    \begin{align*}%
    \Re &: \text{ Jika } &&#1_1 \text{ adalah } #2_1 \text{,} &&\ldots \text{,} &&#1_#5 \text{ adalah } #2_#5,%
    &&\text{maka } &&#3 = \Vec{#4}\cdot \Vec{#1}%
    \end{align*}%
    }%
    }%
}{\iftoggle{tworules}{\iftoggle{oneante}{%
    \vspace*{-\baselineskip}
        \begin{align*}%
        \Re_1 &: \text{ Jika } &&#1 \text{ adalah } #2_1 \text{,} &&\text{maka } &&#3_1 = #4_{1,0}+#4_{1,1}#1\\%
        \Re_2 &: \text{ Jika } &&#1 \text{ adalah } #2_2 \text{,} &&\text{maka } &&#3_2 = #4_{2,0}+#4_{2,1}#1%
        \end{align*}%
    }{\iftoggle{twoante}{%
    \vspace*{-\baselineskip}
        \begin{align*}%
        \Re_1 &: \text{ Jika } &&#1_1 \text{ adalah } #2_{1,1} &&\text{dan} &&#1_2 \text{ adalah } #2_{1,2},%
        &&\text{maka } &&#3_1 = \Vec{#4_1}\cdot \Vec{#1}\\%
        \Re_2 &: \text{ Jika } &&#1_1 \text{ adalah } #2_{2,1} &&\text{dan} &&#1_2 \text{ adalah } #2_{2,2},%
        &&\text{maka } &&#3_2 = \Vec{#4_2}\cdot \Vec{#1}%
        \end{align*}%
    }{\vspace*{-\baselineskip}
        \begin{align*}%
        \Re_1 &: \text{ Jika } &&#1_1 \text{ adalah } #2_{1,1} \text{,} &&\ldots \text{,} &&#1_#5 \text{ adalah } #2_{1,#5} \text{,}
        &&\text{maka } &&#3_1 = \Vec{#4_1}\cdot \Vec{#1}\\%
        \Re_2 &: \text{ Jika } &&#1_1 \text{ adalah } #2_{2,1} \text{,} &&\ldots \text{,} &&#1_#5 \text{ adalah } #2_{2,#5} \text{,}
        &&\text{maka } &&#3_2 = \Vec{#4_2}\cdot \Vec{#1}%
        \end{align*}%
    }%
    }%
}{\iftoggle{threerules}{\iftoggle{oneante}{%
        \vspace*{-\baselineskip}
        \begin{align*}%
        \Re_1 &: \text{ Jika } &&#1 \text{ adalah } #2_1 \text{,} &&\text{maka } &&#3_1 = #4_{1,0}+#4_{1,1}#1\\%
        \Re_2 &: \text{ Jika } &&#1 \text{ adalah } #2_2 \text{,} &&\text{maka } &&#3_2 = #4_{2,0}+#4_{2,1}#1\\%
        \Re_#6 &: \text{ Jika } &&#1 \text{ adalah } #2_#6 \text{,} &&\text{maka } &&#3_#6 = #4_{#6,0}+#4_{#6,1}#1%
        \end{align*}%
    }{\iftoggle{twoante}{%
    \vspace*{-\baselineskip}
        \begin{align*}%
        \Re_1 &: \text{ Jika } &&#1_1 \text{ adalah } #2_{1,1} &&\text{dan} &&#1_2 \text{ adalah } #2_{1,2},%
        &&\text{maka } &&##3_1 = \Vec{#4_1}\cdot \Vec{#1}\\%
        \Re_2 &: \text{ Jika } &&#1_1 \text{ adalah } #2_{2,1} &&\text{dan} &&#1_2 \text{ adalah } #2_{2,2},%
        &&\text{maka } &&#3_2 = \Vec{#4_2}\cdot \Vec{#1}\\%
        \Re_#6 &: \text{ Jika } &&#1_1 \text{ adalah } #2_{#6,1} &&\text{dan} &&#1_2 \text{ adalah } #2_{#6,2},%
        &&\text{maka } &&#3_#6 = \Vec{#4_#6}\cdot \Vec{#1}%
        \end{align*}%
    }{\vspace*{-\baselineskip}
        \begin{align*}%
        \Re_1 &: \text{ Jika } &&#1_1 \text{ adalah } #2_{1,1} \text{,} &&\ldots \text{,} &&#1_#5 \text{ adalah } #2_{1,#5} \text{,}
        &&\text{maka } &&#3_1 = \Vec{#4_1}\cdot \Vec{#1}\\%
        \Re_2 &: \text{ Jika } &&#1_1 \text{ adalah } #2_{2,1} \text{,} &&\ldots \text{,} &&#1_#5 \text{ adalah } #2_{2,#5} \text{,}
        &&\text{maka } &&#3_2 = \Vec{#4_2}\cdot \Vec{#1}\\%
        \Re_#6 &: \text{ Jika } &&#1_1 \text{ adalah } #2_{#6,1} \text{,} &&\ldots \text{,} &&#1_#5 \text{ adalah } #2_{#6,#5} \text{,}
        &&\text{maka } &&#3_#6 = \Vec{#4_#6}\cdot \Vec{#1}%
        \end{align*}%
    }%
    }%
}{\iftoggle{oneante}{%
\vspace*{-\baselineskip}
        \begin{align*}%
        \Re_1 &: \text{ Jika } &&#1 \text{ adalah } #2_1 \text{,} &&\text{maka } &&#3_1 = #4_{1,0}+#4_{1,1}#1\\%
        \Re_2 &: \text{ Jika } &&#1 \text{ adalah } #2_2 \text{,} &&\text{maka } &&#3_2 = #4_{2,0}+#4_{2,1}#1\\%
        & &&\vdots \\
        \Re_#6 &: \text{ Jika } &&#1 \text{ adalah } #2_#6 \text{,} &&\text{maka } &&#3_#6 = #4_{#6,0}+#4_{#6,1}#1%
        \end{align*}%
    }{\iftoggle{twoante}{%
    \vspace*{-\baselineskip}
        \begin{align*}%
        \Re_1 &: \text{ Jika } &&#1_1 \text{ adalah } #2_{1,1} &&\text{dan} &&#1_2 \text{ adalah } #2_{1,2},%
        &&\text{maka } &&#3_1 = \Vec{#4_1}\cdot \Vec{#1}\\%
        \Re_2 &: \text{ Jika } &&#1_1 \text{ adalah } #2_{2,1} &&\text{dan} &&#1_2 \text{ adalah } #2_{2,2},%
        &&\text{maka } &&#3_2 = \Vec{#4_2}\cdot \Vec{#1}\\%
        & && &&\vdots \\
        \Re_#6 &: \text{ Jika } &&#1_1 \text{ adalah } #2_{#6,1} &&\text{dan} &&#1_2 \text{ adalah } #2_{#6,2},%
        &&\text{maka } &&#3_#6 = \Vec{#4_#6}\cdot \Vec{#1}%
        \end{align*}%
    }{\vspace*{-\baselineskip}
        \begin{align*}%
        \Re_1 &: \text{ Jika } &&#1_1 \text{ anggota } #2_{1,1} \text{,} &&\ldots \text{,} &&#1_#5 \text{ anggota } #2_{1,#5} \text{,}
        &&\text{maka } &&#3_1 = \mathbf{#4}_1\cdot \mathbf{\Tilde{#1}}\\%
        \Re_2 &: \text{ Jika } &&#1_1 \text{ anggota } #2_{2,1} \text{,} &&\ldots \text{,} &&#1_#5 \text{ anggota } #2_{2,#5} \text{,}
        &&\text{maka } &&#3_2 = \mathbf{#4}_2\cdot \mathbf{\Tilde{#1}}\\%
        & && &&\vdots \\
        \Re_#6 &: \text{ Jika } &&#1_1 \text{ anggota } #2_{#6,1} \text{,} &&\ldots \text{,} &&#1_#5 \text{ anggota } #2_{#6,#5} \text{,}
        &&\text{maka } &&#3_#6 = \mathbf{#4}_#6\cdot \mathbf{\Tilde{#1}}%
        \end{align*}%
    }%
    }%
}%
}%
}%
}%

%--- MAMDANI FLC
\NewDocumentCommand\flcmamdani{O{x}O{A}O{y}O{B}O{n}O{m}O{a}}{%
\setboolflc{#5}{#6}%
\vspace*{-\baselineskip}
\iftoggle{onerule}{\iftoggle{oneante}{%
        \vspace*{-\baselineskip}
        \begin{alignat*}{4}%
        \Re &: \text{ Jika } &&#1 \text{ adalah } #2 \text{,} &&\text{maka } &&#3 \text{ adalah } #4 \\%
        \text{Fakta} &: &&#1 = #7_1 \\
        \hline
        \text{Konklusi} &: && && &&#3 = #3_0
        \end{alignat*}%
    }{\iftoggle{twoante}{%
    \vspace*{-\baselineskip}
        \begin{align*}%
            \Re &: \text{ Jika } &&#1_1 \text{ adalah } #2_1 &&\text{ dan } &&#1_2 \text{ adalah } #2_2\text{,}%
            &&\text{maka } &&#3 \text{ adalah } #4 \\%
            \text{Fakta} &: &&#1_1 = #7_1 &&\text{ dan } &&#1_2 = #7_2 \\
            \hline
            \text{Konklusi} &: && && && && &&#3 = #3_0
        \end{align*}%
    }{\vspace*{-\baselineskip}
    \begin{align*}%
    \Re &: \text{ Jika } &&#1_1 \text{ adalah } #2_1 \text{,} &&\ldots \text{,} &&#1_#5 \text{ adalah } #2_#5,%
    &&\text{maka } &&#3 \text{ adalah } #4\\%
    \text{Fakta} &: &&#1_1 = #7_1 \text{,} &&\ldots \text{,} &&#1_#5 = #7_#5 \\
    \hline
    \text{Konklusi} &: && && && && &&#3  = #3_0
    \end{align*}%
    }%
    }%
}{\iftoggle{tworules}{\iftoggle{oneante}{%
    \vspace*{-\baselineskip}
        \begin{align*}%
        \Re_1 &: \text{ Jika } &&#1 \text{ adalah } #2_1 \text{,} &&\text{maka } &&#3 \text{ adalah } #4_1\\%
        \Re_2 &: \text{ Jika } &&#1 \text{ adalah } #2_2 \text{,} &&\text{maka } &&#3 \text{ adalah } #4_2\\%
        \text{Fakta} &: &&#1 = #7_1 \\
        \hline
        \text{Konklusi} &: && && &&#3  = #3_0
        \end{align*}%
    }{\iftoggle{twoante}{%
    \vspace*{-\baselineskip}
        \begin{align*}%
        \Re_1 &: \text{ Jika } &&#1_1 \text{ adalah } #2_{1,1} &&\text{dan} &&#1_2 \text{ adalah } #2_{1,2},%
        &&\text{maka } &&#3 \text{ adalah } #4_1\\%
        \Re_2 &: \text{ Jika } &&#1_1 \text{ adalah } #2_{2,1} &&\text{dan} &&#1_2 \text{ adalah } #2_{2,2},%
        &&\text{maka } &&#3 \text{ adalah } #4_2\\%
        \text{Fakta} &: &&#1_1 = #7_1 &&\text{ dan } &&#1_2 = #7_2 \\
        \hline
        \text{Konklusi} &: && && && && &&#3  = #3_0
        \end{align*}%
    }{\vspace*{-\baselineskip}
        \begin{align*}%
        \Re_1 &: \text{ Jika } &&#1_1 \text{ adalah } #2_{1,1} \text{,} &&\ldots \text{,} &&#1_#5 \text{ adalah } #2_{1,#5} \text{,}
        &&\text{maka } &&#3 \text{ adalah } #4_1\\%
        \Re_2 &: \text{ Jika } &&#1_1 \text{ adalah } #2_{2,1} \text{,} &&\ldots \text{,} &&#1_#5 \text{ adalah } #2_{2,#5} \text{,}
        &&\text{maka } &&#3 \text{ adalah } #4_2\\%
        \text{Fakta} &: &&#1_1 = #7_1 \text{,} &&\ldots \text{,} &&#1_#5 = #7_#5 \\
        \hline
        \text{Konklusi} &: && && && && &&#3  = #3_0
        \end{align*}%
    }%
    }%
}{\iftoggle{threerules}{\iftoggle{oneante}{%
        \vspace*{-\baselineskip}
        \begin{align*}%
        \Re_1 &: \text{ Jika } &&#1 \text{ adalah } #2_1 \text{,} &&\text{maka } &&#3 \text{ adalah } #4_1\\%
        \Re_2 &: \text{ Jika } &&#1 \text{ adalah } #2_2 \text{,} &&\text{maka } &&#3 \text{ adalah } #4_2\\%
        \Re_#6 &: \text{ Jika } &&#1 \text{ adalah } #2_#6 \text{,} &&\text{maka } &&#3 \text{ adalah } #4_#6\\%
        \text{Fakta} &: &&#1 = #7_1 \\
        \hline
        \text{Konklusi} &: && && &&#3  = #3_0
        \end{align*}%
    }{\iftoggle{twoante}{%
    \vspace*{-\baselineskip}
        \begin{align*}%
        \Re_1 &: \text{ Jika } &&#1_1 \text{ adalah } #2_{1,1} &&\text{dan} &&#1_2 \text{ adalah } #2_{1,2},%
        &&\text{maka } &&#3 \text{ adalah } #4_1\\%
        \Re_2 &: \text{ Jika } &&#1_1 \text{ adalah } #2_{2,1} &&\text{dan} &&#1_2 \text{ adalah } #2_{2,2},%
        &&\text{maka } &&#3 \text{ adalah } #4_2\\%
        \Re_#6 &: \text{ Jika } &&#1_1 \text{ adalah } #2_{#6,1} &&\text{dan} &&#1_2 \text{ adalah } #2_{#6,2},%
        &&\text{maka } &&#3 \text{ adalah } #4_#6\\%
        \text{Fakta} &: &&#1_1 = #7_1 &&\text{ dan } &&#1_2 = #7_2 \\
        \hline
        \text{Konklusi} &: && && && && &&#3  = #3_0
        \end{align*}%
    }{\vspace*{-\baselineskip}
        \begin{align*}%
        \Re_1 &: \text{ Jika } &&#1_1 \text{ adalah } #2_{1,1} \text{,} &&\ldots \text{,} &&#1_#5 \text{ adalah } #2_{1,#5} \text{,}
        &&\text{maka } &&#3 \text{ adalah } #4_1\\%
        \Re_2 &: \text{ Jika } &&#1_1 \text{ adalah } #2_{2,1} \text{,} &&\ldots \text{,} &&#1_#5 \text{ adalah } #2_{2,#5} \text{,}
        &&\text{maka } &&#3 \text{ adalah } #4_2\\%
        \Re_#6 &: \text{ Jika } &&#1_1 \text{ adalah } #2_{#6,1} \text{,} &&\ldots \text{,} &&#1_#5 \text{ adalah } #2_{#6,#5} \text{,}
        &&\text{maka } &&#3 \text{ adalah } #4_#6\\%
        \text{Fakta} &: &&#1_1 = #7_1 \text{,} &&\ldots \text{,} &&#1_#5 = #7_#5 \\
        \hline
        \text{Konklusi} &: && && && && &&#3 = #3_0
        \end{align*}%
    }%
    }%
}{\iftoggle{oneante}{%
\vspace*{-\baselineskip}
        \begin{align*}%
        \Re_1 &: \text{ Jika } &&#1 \text{ adalah } #2_1 \text{,} &&\text{maka } &&#3 \text{ adalah } #4_1\\%
        \Re_2 &: \text{ Jika } &&#1 \text{ adalah } #2_2 \text{,} &&\text{maka } &&#3 \text{ adalah } #4_2\\%
        & &&\vdots \\
        \Re_#6 &: \text{ Jika } &&#1 \text{ adalah } #2_#6 \text{,} &&\text{maka } &&#3 \text{ adalah } #4_#6\\%
        \text{Fakta} &: &&#1 = #7_1 \\
        \hline
        \text{Konklusi} &: && && &&#3 = #3_0
        \end{align*}%
    }{\iftoggle{twoante}{%
    \vspace*{-\baselineskip}
        \begin{align*}%
        \Re_1 &: \text{ Jika } &&#1_1 \text{ adalah } #2_{1,1} &&\text{dan} &&#1_2 \text{ adalah } #2_{1,2},%
        &&\text{maka } &&#3 \text{ adalah } #4_1\\%
        \Re_2 &: \text{ Jika } &&#1_1 \text{ adalah } #2_{2,1} &&\text{dan} &&#1_2 \text{ adalah } #2_{2,2},%
        &&\text{maka } &&#3 \text{ adalah } #4_2\\%
        & && &&\vdots \\
        \Re_#6 &: \text{ Jika } &&#1_1 \text{ adalah } #2_{#6,1} &&\text{dan} &&#1_2 \text{ adalah } #2_{#6,2},%
        &&\text{maka } &&#3 \text{ adalah } #4_#6\\%
        \text{Fakta} &: &&#1_1 = #7_1 &&\text{ dan } &&#1_2 = #7_2 \\
        \hline
        \text{Konklusi} &: && && && && &&#3 = #3_0
        \end{align*}%
    }{\vspace*{-\baselineskip}
        \begin{align*}%
        \Re_1 &: \text{ Jika } &&#1_1 \text{ anggota } #2_{1,1} \text{,} &&\ldots \text{,} &&#1_#5 \text{ anggota } #2_{1,#5} \text{,}
        &&\text{maka } &&#3 \text{ anggota } #4_1\\%
        \Re_2 &: \text{ Jika } &&#1_1 \text{ anggota } #2_{2,1} \text{,} &&\ldots \text{,} &&#1_#5 \text{ anggota } #2_{2,#5} \text{,}
        &&\text{maka } &&#3 \text{ anggota } #4_2\\%
        & && &&\vdots \\
        \Re_#6 &: \text{ Jika } &&#1_1 \text{ anggota } #2_{#6,1} \text{,} &&\ldots \text{,} &&#1_#5 \text{ anggota } #2_{#6,#5} \text{,}
        &&\text{maka } &&#3 \text{ anggota } #4_#6\\%
        \text{Fakta} &: &&#1_1 = #7_1 \text{,} &&\ldots \text{,} &&#1_#5 = #7_#5 \\
        \hline
        \text{Konklusi} &: && && && && &&#3 = #3_0
        \end{align*}%
    }%
    }%
}%
}%
}%
}%

%---TSK FUZZY RULES
\NewDocumentCommand\flctsk{O{x}O{A}O{y}O{b}O{n}O{m}O{a}}{%
\setboolflc{#5}{#6}%
\vspace*{-\baselineskip}
\iftoggle{onerule}{\iftoggle{oneante}{%
        \vspace*{-\baselineskip}
        \begin{alignat*}{4}%
        \Re &: \text{ Jika } &&#1 \text{ adalah } #2 \text{,} &&\text{maka } &&#3 = #4_0 + #4_1#1 \\%
        \text{Fakta} &: &&#1 = #7_1 \\
        \hline
        \text{Konklusi} &: && && &&#3 = #3_0
        \end{alignat*}%
    }{\iftoggle{twoante}{%
    \vspace*{-\baselineskip}
        \begin{align*}%
            \Re &: \text{ Jika } &&#1_1 \text{ adalah } #2_1 &&\text{ dan } &&#1_2 \text{ adalah } #2_2\text{,}%
            &&\text{maka } &&#3 = \Vec{#4}\cdot \Vec{#1} \\%
            \text{Fakta} &: &&#1_1 = #7_1 &&\text{ dan } &&#1_2 = #7_2 \\
            \hline
            \text{Konklusi} &: && && && && &&#3 = #3_0
        \end{align*}%
    }{\vspace*{-\baselineskip}
    \begin{align*}%
    \Re &: \text{ Jika } &&#1_1 \text{ adalah } #2_1 \text{,} &&\ldots \text{,} &&#1_#5 \text{ adalah } #2_#5,%
    &&\text{maka } &&#3 = \Vec{#4}\cdot \Vec{#1}\\%
    \text{Fakta} &: &&#1_1 = #7_1 \text{,} &&\ldots \text{,} &&#1_#5 = #7_#5 \\
    \hline
    \text{Konklusi} &: && && && && &&#3  = #3_0
    \end{align*}%
    }%
    }%
}{\iftoggle{tworules}{\iftoggle{oneante}{%
    \vspace*{-\baselineskip}
        \begin{align*}%
        \Re_1 &: \text{ Jika } &&#1 \text{ adalah } #2_1 \text{,} &&\text{maka } &&#3_1 = #4_{1,0}+#4_{1,1}#1\\%
        \Re_2 &: \text{ Jika } &&#1 \text{ adalah } #2_2 \text{,} &&\text{maka } &&#3_2 = #4_{2,0}+#4_{2,1}#1\\%
        \text{Fakta} &: &&#1 = #7_1 \\
        \hline
        \text{Konklusi} &: && && &&#3  = #3_0
        \end{align*}%
    }{\iftoggle{twoante}{%
    \vspace*{-\baselineskip}
        \begin{align*}%
        \Re_1 &: \text{ Jika } &&#1_1 \text{ adalah } #2_{1,1} &&\text{dan} &&#1_2 \text{ adalah } #2_{1,2},%
        &&\text{maka } &&#3_1 = \Vec{#4_1}\cdot \Vec{#1}\\%
        \Re_2 &: \text{ Jika } &&#1_1 \text{ adalah } #2_{2,1} &&\text{dan} &&#1_2 \text{ adalah } #2_{2,2},%
        &&\text{maka } &&#3_2 = \Vec{#4_2}\cdot \Vec{#1}\\%
        \text{Fakta} &: &&#1_1 = #7_1 &&\text{ dan } &&#1_2 = #7_2 \\
        \hline
        \text{Konklusi} &: && && && && && #3  = #3_0
        \end{align*}%
    }{\vspace*{-\baselineskip}
        \begin{align*}%
        \Re_1 &: \text{ Jika } &&#1_1 \text{ adalah } #2_{1,1} \text{,} &&\ldots \text{,} &&#1_#5 \text{ adalah } #2_{1,#5} \text{,}
        &&\text{maka } &&#3_1 = \Vec{#4_1}\cdot \Vec{#1}\\%
        \Re_2 &: \text{ Jika } &&#1_1 \text{ adalah } #2_{2,1} \text{,} &&\ldots \text{,} &&#1_#5 \text{ adalah } #2_{2,#5} \text{,}
        &&\text{maka } &&#3_2 = \Vec{#4_2}\cdot \Vec{#1}\\%
        \text{Fakta} &: &&#1_1 = #7_1 \text{,} &&\ldots \text{,} &&#1_#5 = #7_#5 \\
        \hline
        \text{Konklusi} &: && && && && &&#3  = #3_0
        \end{align*}%
    }%
    }%
}{\iftoggle{threerules}{\iftoggle{oneante}{%
        \vspace*{-\baselineskip}
        \begin{align*}%
        \Re_1 &: \text{ Jika } &&#1 \text{ adalah } #2_1 \text{,} &&\text{maka } &&#3_1 = #4_{1,0}+#4_{1,1}#1\\%
        \Re_2 &: \text{ Jika } &&#1 \text{ adalah } #2_2 \text{,} &&\text{maka } &&#3_2 = #4_{2,0}+#4_{2,1}#1\\%
        \Re_#6 &: \text{ Jika } &&#1 \text{ adalah } #2_#6 \text{,} &&\text{maka } &&#3_#6 = #4_{#6,0}+#4_{#6,1}#1\\%
        \text{Fakta} &: &&#1 = #7_1 \\
        \hline
        \text{Konklusi} &: && && &&#3  = #3_0
        \end{align*}%
    }{\iftoggle{twoante}{%
    \vspace*{-\baselineskip}
        \begin{align*}%
        \Re_1 &: \text{ Jika } &&#1_1 \text{ adalah } #2_{1,1} &&\text{dan} &&#1_2 \text{ adalah } #2_{1,2},%
        &&\text{maka } &&##3_1 = \Vec{#4_1}\cdot \Vec{#1}\\%
        \Re_2 &: \text{ Jika } &&#1_1 \text{ adalah } #2_{2,1} &&\text{dan} &&#1_2 \text{ adalah } #2_{2,2},%
        &&\text{maka } &&#3_2 = \Vec{#4_2}\cdot \Vec{#1}\\%
        \Re_#6 &: \text{ Jika } &&#1_1 \text{ adalah } #2_{#6,1} &&\text{dan} &&#1_2 \text{ adalah } #2_{#6,2},%
        &&\text{maka } &&#3_#6 = \Vec{#4_#6}\cdot \Vec{#1}\\%
        \text{Fakta} &: &&#1_1 = #7_1 &&\text{ dan } &&#1_2 = #7_2 \\
        \hline
        \text{Konklusi} &: && && && && &&#3  = #3_0
        \end{align*}%
    }{\vspace*{-\baselineskip}
        \begin{align*}%
        \Re_1 &: \text{ Jika } &&#1_1 \text{ adalah } #2_{1,1} \text{,} &&\ldots \text{,} &&#1_#5 \text{ adalah } #2_{1,#5} \text{,}
        &&\text{maka } &&#3_1 = \Vec{#4_1}\cdot \Vec{#1}\\%
        \Re_2 &: \text{ Jika } &&#1_1 \text{ adalah } #2_{2,1} \text{,} &&\ldots \text{,} &&#1_#5 \text{ adalah } #2_{2,#5} \text{,}
        &&\text{maka } &&#3_2 = \Vec{#4_2}\cdot \Vec{#1}\\%
        \Re_#6 &: \text{ Jika } &&#1_1 \text{ adalah } #2_{#6,1} \text{,} &&\ldots \text{,} &&#1_#5 \text{ adalah } #2_{#6,#5} \text{,}
        &&\text{maka } &&#3_#6 = \Vec{#4_#6}\cdot \Vec{#1}\\%
        \text{Fakta} &: &&#1_1 = #7_1 \text{,} &&\ldots \text{,} &&#1_#5 = #7_#5 \\
        \hline
        \text{Konklusi} &: && && && && &&#3 = #3_0
        \end{align*}%
    }%
    }%
}{\iftoggle{oneante}{%
\vspace*{-\baselineskip}
        \begin{align*}%
        \Re_1 &: \text{ Jika } &&#1 \text{ adalah } #2_1 \text{,} &&\text{maka } &&#3_1 = #4_{1,0}+#4_{1,1}#1\\%
        \Re_2 &: \text{ Jika } &&#1 \text{ adalah } #2_2 \text{,} &&\text{maka } &&#3_2 = #4_{2,0}+#4_{2,1}#1\\%
        & &&\vdots \\
        \Re_#6 &: \text{ Jika } &&#1 \text{ adalah } #2_#6 \text{,} &&\text{maka } &&#3_#6 = #4_{#6,0}+#4_{#6,1}#1\\%
        \text{Fakta} &: &&#1 = #7_1 \\
        \hline
        \text{Konklusi} &: && && &&#3 = #3_0
        \end{align*}%
    }{\iftoggle{twoante}{%
    \vspace*{-\baselineskip}
        \begin{align*}%
        \Re_1 &: \text{ Jika } &&#1_1 \text{ adalah } #2_{1,1} &&\text{dan} &&#1_2 \text{ adalah } #2_{1,2},%
        &&\text{maka } &&#3_1 = \Vec{#4_1}\cdot \Vec{#1}\\%
        \Re_2 &: \text{ Jika } &&#1_1 \text{ adalah } #2_{2,1} &&\text{dan} &&#1_2 \text{ adalah } #2_{2,2},%
        &&\text{maka } &&#3_2 = \Vec{#4_2}\cdot \Vec{#1}\\%
        & && &&\vdots \\
        \Re_#6 &: \text{ Jika } &&#1_1 \text{ adalah } #2_{#6,1} &&\text{dan} &&#1_2 \text{ adalah } #2_{#6,2},%
        &&\text{maka } &&#3_#6 = \Vec{#4_#6}\cdot \Vec{#1}\\%
        \text{Fakta} &: &&#1_1 = #7_1 &&\text{ dan } &&#1_2 = #7_2 \\
        \hline
        \text{Konklusi} &: && && && && &&#3 = #3_0
        \end{align*}%
    }{\vspace*{-\baselineskip}
        \begin{align*}%
        \Re_1 &: \text{ Jika } &&#1_1 \text{ anggota } #2_{1,1} \text{,} &&\ldots \text{,} &&#1_#5 \text{ anggota } #2_{1,#5} \text{,}
        &&\text{maka } &&#3 = \mathbf{#4}_1\cdot \mathbf{\Tilde{#1}}\\%
        \Re_2 &: \text{ Jika } &&#1_1 \text{ anggota } #2_{2,1} \text{,} &&\ldots \text{,} &&#1_#5 \text{ anggota } #2_{2,#5} \text{,}
        &&\text{maka } &&#3 = \mathbf{#4}_2\cdot \mathbf{\Tilde{#1}}\\%
        & && &&\vdots \\
        \Re_#6 &: \text{ Jika } &&#1_1 \text{ anggota } #2_{#6,1} \text{,} &&\ldots \text{,} &&#1_#5 \text{ anggota } #2_{#6,#5} \text{,}
        &&\text{maka } &&#3 = \mathbf{#4}_#6\cdot \mathbf{\Tilde{#1}}\\%
        \text{Fakta} &: &&#1_1 = #7_1 \text{,} &&\ldots \text{,} &&#1_#5 = #7_#5 \\
        \hline
        \text{Konklusi} &: && && && && &&#3 = #3_0
        \end{align*}%
    }%
    }%
}%
}%
}%
}%
\makeatother